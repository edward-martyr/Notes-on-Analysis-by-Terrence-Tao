\section{Appendix: the basics of mathematical logic}
\subsection{Notes}
\subsubsection{Theorems}

\subsubsection{Remarks}
I started reading Appendix A realising my lack of sophistication with Logic after finishing Chapter 2.

Discussing cases in a proof is a common example of using vacuously true implications for a non-trivial result, e.g., if we want to prove that $P(x)$ is true, we can prove the implications that if $x$ is even, then $P(x)$ is true and that if $x$ is odd, then $P(x)$ is true, even if one implication must have a false hypothesis because $x$ cannot be both even and false and thus be vacuous.

\subsection{Practices}
\paragraph{Exercise A.1.1.} Both $X$ and $Y$ are true, or both are false.

\paragraph{Exercise A.1.2.} Either X is true, or Y is true, but not both.

\paragraph{Exercise A.1.3.} Yes, because they are \emph{equally} true or \emph{equally} false, and they can only be true or false.

\paragraph{Exercise A.1.4.} No, because that $Y$ is true does not necessarily mean that $X$ is true. For example, $X$ is ``$a=3$''; $Y$ is ``$a^2=9$''.

\paragraph{Exercise A.1.5.} Yes, because if $X$ is true, then $Y$ is true, then $Z$ is true; if $Z$ is true, then $Y$ is true, then $X$ is true. They are equally true or equally false.

\paragraph{Exercise A.1.6.} Yes, likewise.

