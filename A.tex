\section{Appendix: the basics of mathematical logic}
\subsection{Notes}
\subsubsection{Theorems}
\begin{axiom}[Reflexive axiom]
    Given any object $x$, we have $x = x$.
\end{axiom}
\begin{axiom}[Symmetry axiom]
    Given any two objects $x$ and $y$ of the same
type, if $x = y$, then $y = x$.
\end{axiom}
\begin{axiom}[Transitive axiom]
    Given any three objects $x, y, z$ of the same
type, if $x = y$ and $y = z$, then $x = z$.
\end{axiom}
\begin{axiom}[Substitution axiom]
    Given any two objects $x$ and $y$ of the same type, if $x = y$, then $f(x) = f(y)$ for all functions or operations $f$. Similarly, for any property $P(x)$ depending on $x$, if $x = y$, then $P(x)$ and $P(y)$ are equivalent statements.    
\end{axiom}

\subsubsection{Remarks}
I started reading Appendix A realising my lack of sophistication with Logic after finishing Chapter 2.

Discussing cases in a proof is a common example of using vacuously true implications for a non-trivial result, e.g., if we want to prove that $P(x)$ is true for some integer $x$, we can prove the implications that if $x$ is even, then $P(x)$ is true and that if $x$ is odd, then $P(x)$ is true, even if one implication must have a false hypothesis because $x$ cannot be both even and false and thus be vacuous.

Equality is also worth reviewing. I was indecisive whether it was legitimate to add a number to both sides of an equality when doing the exercises in Chapter 2. It actually follows the \emph{substitution axiom} of equality.

\subsection{Practices}
\paragraph{Exercise A.1.1.} Both $X$ and $Y$ are true, or both are false.

\paragraph{Exercise A.1.2.} Either $X$ is true, or $Y$ is true, but not both.

\paragraph{Exercise A.1.3.} Yes, because they are \emph{equally} true or \emph{equally} false, and they can only be true or false.

\paragraph{Exercise A.1.4.} No, because that $Y$ is true does not necessarily mean that $X$ is true. For example, $X$ is ``$a=3$''; $Y$ is ``$a^2=9$''.

\paragraph{Exercise A.1.5.} Yes, because if $X$ is true, then $Y$ is true, then $Z$ is true; if $Z$ is true, then $Y$ is true, then $X$ is true. They are equally true or equally false.

\paragraph{Exercise A.1.6.} Yes, likewise.

\paragraph{Exercise A.5.1.}
\begin{enumerate}
    \item $\iff \forall (x,y) \in (\mathbb{R}^+)^2  : y^2=x$, which is a false statement (e.g. $x=1,y=2$). 
    \item $\iff \exists x \in \mathbb{R}^+ \quad \forall y \in \mathbb{R}^+ : y^2=x$, which is a false statement. \begin{proof}
        For the sake of contradiction, suppose that the statement holds for some $x \in \mathbb{R}^+$; so $y^2=x$. But for all $y \in \mathbb{R}^+$, $y^2=x$, so $(y+1)^2=x$ where $y+1 \in \mathbb{R}^+$; so $y^2=x-(2y+1)=x$; so $2y+1=0$. But $2y+1$ is positive, so there is a contradiction.
    \end{proof}
    \item $\iff \exists (x,y) \in (\mathbb{R}^+)^2  : y^2=x$, which is a true statement (e.g. $x=1,y=1$). 
    \item $\iff \forall y \in \mathbb{R}^+ \quad \exists x \in \mathbb{R}^+ : y^2=x$, which is a true statement. \begin{proof}
        Let $a := {y}^2$, so $a\in \mathbb{R}^+$ because $y \in \mathbb{R}^+$. Thus, there exists some $x=a\in \mathbb{R}^+$ which satisfies the statement.
    \end{proof}
    \item $\iff \exists y \in \mathbb{R}^+ \quad \forall x \in \mathbb{R}^+ : y^2=x$, which is a false statement (likewise).
\end{enumerate}

\paragraph{Exercise A.7.1.}
\begin{proof}
    Given $a=b$ and $c = d$,
    \begin{align*}
        \text{we have } {} &b=a && \text{(Symmetry axiom)}\\
        \implies {} &b+c=a+c && \text{(Substitution axiom)}\\
        \text{but } {} &a=a && \text{(Reflexive axiom)}\\
        \implies {} &a+c=a+c && \text{(Substitution axiom)}\\
        \implies {} &a+c=a+d && \text{(Substitution axiom)}\\
        \text{so } {} &b+c=a+d && \text{(Transitive axiom)}\\
        \implies {} &a+d=b+c. && \text{(Symmetry axiom)} \qedhere
    \end{align*}
\end{proof}